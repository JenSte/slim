\documentclass[a4paper,10pt]{article}
\usepackage{calc}
\usepackage{color}
\usepackage{epsfig}
\usepackage{lastpage}
\usepackage{listings}
\usepackage{url}
\usepackage{xspace}

\newenvironment{Description}[1][\quad]{%
  \begin{list}{}{%
      \renewcommand{\makelabel}[1]{\textbf{##1}\hfill}%
      \settowidth{\labelwidth}{\textbf{#1}}%
      \setlength{\leftmargin}{\labelwidth+\labelsep}%
  }%
}{%
  \end{list}%
}

\newenvironment{MyTable}{
 \bigskip
 \begin{table}[!hbp]\begin{center}
} {
 \end{center}\end{table}
}
\newenvironment{MyFigure}{
 \begin{figure}[!ptb]\begin{center}
} {
 \end{center}\end{figure}
}
\newenvironment{MyFigureHere}{
 \begin{figure}[!hbp]\begin{center}
} {
 \end{center}\end{figure}
}
\newcommand{\slim}{\textsc{Slim}\xspace}
\newcommand{\fw}{\tt\bf}

\begin{document}

\title{
  The Simple Linux Image Maker
}
\author{
  Richard Cochran \\
  OMICRON electronics GmbH \\
  {\tt richard.cochran@omicron.at}
}
\date{June 9, 2011}
\maketitle
\newpage
\tableofcontents
\newpage

\definecolor{gray}{rgb}{0.75,0.75,0.75}
\lstset{
  aboveskip=\baselineskip,
  basicstyle=\ttfamily,
  frame=shadowbox,
  framexbottommargin=1em,
  framexleftmargin=1mm,
  framextopmargin=1em,
  language=Make,
  rulesepcolor=\color{gray}
}
\setlength{\parindent}{0pt}
\setlength{\parskip}{\baselineskip}

\section{Introduction}

    The Simple Linux Image Maker (\slim) is a tool to develop root
    file systems for embedded computers running the GNU/Linux
    operating system. The \slim system enables a developer working on
    a Debian GNU/Linux work station to painlessly create a bootable
    image for one or more embedded devices.

    The \slim build system came about during a development project at
    OMICRON electronics which required creating a GNU/Linux based
    embedded computer. After lengthy trials with existing open source
    build systems, it became apparent that no single existing system
    met all of our needs.

\subsection{Audience}

    \slim can be used by two different classes of end users, namely
    build managers and platform integration developers. For the build
    managers, using \slim requires only basic knowledge of working
    with a GNU/Linux system from the command line. To perform a build,
    you need to set three environment variables and then run ``make.''

    The integration developer should posses a good understanding of
    GNU makefiles and should have experience in configuring and
    building software packages directly from source. In addition,
    experience working with the git version control system is required
    to benefit from the advantages of the \slim system.

\subsection{Design Goals}

\begin{enumerate}
\item Simple, transparent build system.
\item Automatically creates manifests and source archives for GPL/OSS
  license compliance.
\item Uses an external cross compiler and glibc \\
  (optionally can use the host compiler).
\item Does not build any host tools, but runs under Debian GNU/Linux.
\item Written completely in GNU makefiles, with minimal bash shell scripting
(no Perl, Python, or other scripting languages).
\item Creates a monolithic image \\
  (does not ``nickel and dime'' with individual packages).
\item Supports multiple architectures.
\item Supports multiple build configurations simultaneously.
\item Optionally builds an external Linux kernel.
\item Supports fetching upstream sources via git or wget.
\end{enumerate}

\subsection{Putting Patches in Their Place.}

    One critical weakness of existing build systems is their
    management of patches to upstream sources. Many upstream software
    packages do not build correctly when cross compiling, and so the
    integration developer must provide patches to these
    packages. Often, the only minimal changes are required, but still,
    for whatever reason,%
    \footnote {
      Perhaps the embedded developer is too lazy to submit his patch,
      or upstream maintainer is not interested in the needs of the
      embedded developer, or maybe a bit of both.
    }
    these patches are not integrated upstream. 

    As a result the integration developer must maintain the
    {\it patches themselves}, along with the build system, in his
    version control system (VCS). In many cases, for example when
    working with the Linux kernel, the only practical way to create
    and maintain the patches is to use the VCS of the upstream
    project. As a result, the developer is faced with the tedious and
    error-prone task of synchronizing his patches within the build
    system to multiple external VC systems.

    The \slim build system presents a novel solution to this
    problem. By supporting fetching of git sources by named commit,
    tag or branch, \slim eliminates the need to duplicate the embedded
    patches within the \slim system itself. Instead, the integration
    developer maintains the patches in an external git repository,
    which is the ``right way'' to do it. Using git, the task of
    tracking external projects, even those in SVN or CVS, is much more
    manageable than with any other method.

    Of course, credit for the fact that the embedded developer can
    simplify his work in this way belongs to git. The \slim build
    system just takes advantage of the power of git. For example, when
    the \slim makefiles contain references to specific commits, a
    ``stable release'' is built. In contrast, when the makefiles
    contain reference to branch heads, a ``bleeding edge'' image can
    be automatically built, without any manual updating of the
    sources.

\subsection{Comparison to Existing Systems}

    There are many different Open Source systems for creating embedded
    systems. Each one has its own purpose and methods, and every
    project has its strengths and weaknesses. \slim owes its existence
    to and stands on the shoulders of a number of them, by both
    stealing the good ideas and avoiding the ugly stuff.

    The following evaluations should be read with a grain salt. The
    facts and opinions expressed date primarily from experience
    gathered during 2009. Open source projects often grow, improve,
    and change very quickly, and so it is well worth the effort to
    take another look at the current offerings now and again.

    With so many choices available, we recommend the following method
    for picking a build system. Using each system in turn, try to
    build and boot a simple image for your own board, including at
    least one custom package that you yourself add. If you need to
    spend more than one week on it, then probably that build system is
    not for you.

\subsubsection{Linux Target Image Builder}

    The Linux Target Image Builder~\cite{ltib} (LTIB) is written in
    Perl and uses RPM as the package format. The documentation is
    adequate, and adding a new package is fairly straightforward.
    Building with LTIB requires root permissions, and adding new
    sources or patches requires placing them in globally shared
    directory under {\fw /opt}.
%
    Patch management is done using a primitive RPM method. LTIB does
    not build a cross toolchain, but it does build many host tools
    which are already available on a Debian system.
%
    LTIB allows building an external Linux kernel, but in practice
    this method is brittle and error-prone, like many of the
    configuration options. One major drawback of LTIB is the fact that
    while the most recent development occurs in an independent open
    source project, essential configuration files and patches for many
    boards are only found in out of date releases from Freescale.
    LTIB does not support fetching sources directly from a VCS.

\subsubsection{Open Embedded}

    Open Embedded~\cite{oe} (OE) is a humongous project based on the
    ``object oriented'' bit bake build tool, written in the python
    scripting language. OE requires the developer to write ``recipes''
    in python, and these recipes can ``inherit'' from base classes.
    The documentation on how to write a recipe and on the many magic
    variable names is very thin indeed. OE supports multiple
    architectures and board types, but the documentation on how to add
    a new board is lacking important detail.%
    \footnote {
      This author freely admits that, despite weeks of effort, he was
      not even able to add a custom package to OE, let alone create a
      new platform type. Maybe he is just too dimwitted to understand
      the OE system.
    }
    An OE build is controlled by means of a host of different
    configuration files.  OE builds the cross toolchain with the glibc
    library, and it also builds many host tools which are already
    available on a Debian system. The OE build produces packages in
    the ipkg format.  One positive aspect of OE is that is can fetch
    sources directly from configured version control systems.

\subsubsection{OpenWrt}

    The OpenWrt~\cite{openwrt} build system is written in GNU
    makefiles, with a bit of Perl and shell scripting. Adding packages
    and kernel modules is adequately documented and is accomplished by
    means of slightly non-standard GNU makefiles. OpenWrt always
    builds its own toolchain and kernel. OpenWrt produces packages in
    the opkg format. The main drawbacks of OpenWrt are that it
    currently only supports uClibc, it is not possible to use an
    external toolchain, and the system does not provide a license
    manifest. At least for the Linux kernel, OpenWrt can fetch sources
    directly from a VCS.

\subsubsection{Others}

\begin{Description}[buildroot0000]

\item[Andriod]
  is a java platform with a stripped down BSD C library.

\item[ptxdist \cite{ptxdist}]
  is written in shell script and makefiles, can compile its own
  toolchain or use an external one, does not support fetching from
  VCS, fairly easy to add a package.

\item[buildroot \cite{buildroot}]
  is makefile based, uses Kconfig, easy to add a package, includes
  complete source tree for all components, uClibc only.

\item[emdebian \cite{emdebian}]
  is an attempt to adapt the Debian system to the needs of embedded
  system development.

\end{Description}

\section{Installing \slim}

\subsection{Prerequisites}

\subsubsection{Cross compiler}

    The fact that \slim does not compile its own toolchain is a
    deliberate design decision.  Choosing the right compiler is very
    important for the success of an embedded project.  Not every
    combination of gcc, glibc, and binutils yields a correctly working
    compiler.  Our recommendation is to use a cross compiler for your
    platform that has been used and tested by as many other people as
    possible.  Often, the compiler provided by the chip vendor is a
    good, safe choice.

    If you find yourself in the position of having to compile your own
    toolchain, then are specialized open source tools to accomplish
    this.
    The original crosstool project~\cite{kegel} is a bit out of date
    by now, but still can be used to produce a working 3.x gcc
    version.
    The crosstool-NG project~\cite{morin} has extended the original
    crosstool idea and provides an easy to use Kconfig (ncurses blue
    screen) menu system to choose the settings and options for your
    toolchain.

    The installation of a cross compiler is, to quote a favorite
    international standard of ours, a ``local issue.''  Every single
    toolchain seems to have its own idea of where it should be
    installed.  Before you begin working with \slim, you should add
    the cross compiler tools into your shell's search path.

\subsubsection{Host tools}

    \slim assumes a Debian GNU/Linux development environment, although
    it is probably possible to use it under other GNU/Linux
    distributions as well. Before using \slim for the first time, you
    should make sure the following Debian packages are installed.

\begin{verbatim}
autoconf          findutils         intltool          mtd-utils
automake          flex              libc6             p7zip-full
autotools-dev     g++               libc6-dev         patch
bash              gawk              libglib2.0-dev    sed
binutils          gcc               liblzo2-2         tar
bison             genext2fs         liblzo2-dev       tftp
bsdmainutils      genisoimage       libncurses5       tftpd
bzip2             git-core          libncurses5-dev   uboot-mkimage
coreutils         git-gui           libtool           unzip
cpp               gitk              linux-libc-dev    util-linux
diff              grep              m4                wget
diffstat          groff             make              zip
e2fslibs          groff-base        mktemp            zlib1g
e2fsprogs         gzip              mount             zlib1g-dev
file              initramfs-tools
\end{verbatim}

\subsection{Installation}

    The \slim build system is managed under the git version control
    system. To install \slim, run the following command.

\begin{lstlisting}
git clone git://github.com/richardcochran/slim.git
\end{lstlisting}

    This will create a new directory called {\fw slim} in your current
    working directory. Once you change directory into {\fw slim}, you
    will be ready to get to work.

\section{Building an Image}

\subsection{Setting the Environment}

    Before you build, you need to set two environment variables,
    {\fw BOARD} and {\fw CROSS\_COMPILE}, described below.
    These variables are used by \slim during the build process.  In
    addition to these two variables, you should also add the path to
    your cross compiler to the {\fw PATH} variable, if it is not
    already in the path.

    Depending on your larger development environment, you may want to
    set the optional variables {\fw SLIM\_GIT} and {\fw SLIM\_WGET}.
    These come into play when using a centralized server in a
    corporate LAN setting, for example.

    \begin{Description}[SLIM\_WGETX]

    \item[BOARD]
      This identifies your target platform. This variable must match
      one of the subdirectories of {\fw slim/config}.

    \item[CROSS\_COMPILE]\mbox{}\\
      This identifies the compiler to use. Please note that the value
      for this variable should end with a dash (or minus sign).

    \item[SLIM\_GIT]
      This names the source URL for those packages which come from a
      git repository. If set, the git sources will be fetched from
      this URL instead of the ones named in the package makefiles.
      The variable can be set to either a local directory or to a file
      server in your LAN. When using a file server, it must accept the
      ``git archive'' command.  If unset, \slim will look for a
      directory called {\fw slim\_git} in your home directory.

    \item[SLIM\_WGET]
      This variable names a proxy web server. If set, the tar files
      will be fetched from this URL instead of the ones named in the
      package makefiles. This is useful when setting up an in-house
      archive of all the source packages used in an image.

    \end{Description}

    Typing these variables into the shell every time you sit down to
    work soon becomes tedious. Instead, you can place them into a
    small shell script, as shown.

\begin{lstlisting}[language=bash,escapechar=X]
# env.sh - Sample settings for the SLIM build system.
export BOARD=nslu
export CROSS_COMPILE=armeb-unknown-linux-gnu-
export PATH=X\$XPATH:/opt/xcc/armeb-unknown-linux-gnu/bin
\end{lstlisting}

    The \slim build system already includes a default {\fw env.sh}
    file which contains the PATH variables for cross compilers shipped
    by the vendor with various development boards. Once you have
    created or customized such a file, then you can simply source it
    before you start a build, as shown:

\begin{lstlisting}[language=bash,escapechar=X]
X\$X . scripts/env.sh
\end{lstlisting}

    If you do not set {\fw SLIM\_GIT} (which is normally the case when
    first starting out), then your need to create a directory to hold
    the git repositories, as shown below.

\begin{lstlisting}[language=bash,escapechar=X]
X\$X mkdir ~/slim_git
\end{lstlisting}

\subsection{Building}

    To build your project, simply type ``make,'' as shown below.

\begin{lstlisting}[language=bash,escapechar=X]
X\$X make
\end{lstlisting}

    Once the build completes, the build products may be found in a
    subdirectory called {\fw slim/\$BOARD}. Normally, the build system
    only prints a summary of the build process, along with any
    compiler warnings or errors. If you encounter an error, or if you
    are just curious, you can enable a verbose build by settings the
    {\fw V} and {\fw Q} variables, as shown.

\begin{lstlisting}[language=bash,escapechar=X]
X\$X make V=1 Q=
\end{lstlisting}

    To completely remove all of the build products, make the
    ``distclean'' target, as shown below.

\begin{lstlisting}[language=bash,escapechar=X]
X\$X make distclean
\end{lstlisting}

\subsection{Rebuilding a Package}

    To force a rebuild of a certain package, type the following
    command. For git-based packages configured to a named branch, this
    will also update your sources to the current version from the
    source repository.

\begin{lstlisting}[language=bash,escapechar=X]
X\$X make pkg/bob/distclean
X\$X make
\end{lstlisting}

\section{Adding a New Package}

    One of the main design goals of \slim is to make adding a custom
    package as easy as possible. \slim makefiles are really just plain
    old GNU makefiles. If you know how to write a makefile, then
    adding a package to \slim should be a breeze. Also, adding third
    party packages is relatively painless. If you can configure and
    compile a package by hand, then it is a fairly straightforward
    exercise to enter those shell commands into a \slim makefile.

\subsection{Creating the makefile}\label{MakeNew}

    Every \slim package has its own directory under the {\fw slim/pkg}
    directory. The first step in adding a new package is to copy a
    template makefile into a new sub-directory of {\fw slim/pkg}. The
    ``new\_package'' make target does this for you automatically. For a
    package named ``bob,'' enter the following command to create the
    initial package.

\begin{lstlisting}[language=bash,escapechar=X]
X\$X make new_package PKG=bob GET=wget AC=no
\end{lstlisting}

    For a git-based package, use this command instead.

\begin{lstlisting}[language=bash,escapechar=X]
X\$X make new_package PKG=bob GET=git AC=no
\end{lstlisting}

    If the package uses the GNU autoconf system, then you should use
    ``AC=yes'' on the command line.

\subsection{Customizing the makefile}

    After creating your package's makefile, the next step is to
    properly customize its variables and rules. As discussed in
    Section~\ref{RequiredTargets}, the targets must not be changed,
    but the rules may be freely changed to do the ``right thing'' for
    your software package. Some of the variables in the template
    makefile are used by the build system for license compliance,
    figuring package dependencies, and for the menu system. These
    variables are not strictly required by the build system, and if
    omitted, \slim tries to fall back to a reasonable default. You are
    free to add any other variables that you need in order to program
    your makefile correctly or to make it more readable.

    \begin{Description}[XXXXXX]
    \item[VER]
      The version string for your package. In the case of a git
      package, this can be a branch name, a tag, or a SHA1 sum.

    \item[GET]
      How to fetch your package's sources, either ``git'' or ``wget.''

    \item[URL]
      The URL for the package's sources. When used as an argument to
      the {\fw fetch} script, the exact form to use differs depending
      on whether {\fw GET} is set to ``git'' or ``wget.''  For git,
      the URL must include the entire path. For wget, the URL should
      not include the file name itself.

    \item[TGZ]
      The file name of the source archive. For git packages, this may
      be different from the name of the repository.

    \end{Description}

\subsubsection{Variables for License Compliance}

    \begin{Description}[UPSTREAMXX]
    \item[LICENSE]
      The type of license. Some examples are: \\
      BSD, GPL, GPL2, GPL3, LGPL, LGPL2, LGPL3, MIT, Proprietary.

    \item[LICFILE]
      The path within the source archive to the file containing the
      text of the license. If this variable is non-empty, then the
      license will be added into {\fw \$BOARD/redist} for the purpose
      of license notification. This should be set for packages under a
      GPL or BSD style license.

    \item[UPSTREAM]
      If you are delivering a modified version of a package under the
      GPL, and the package is in git, then you can use this variable
      to identify the origin version. When you pass this to the
      {\fw fetch} script, it will automatically provide the {\fw diff}
      file required for license compliance.

    \item[BINARIES]
      Identifies which executables the package installs into the
      image. The linkage report uses this list to show the libraries
      to which proprietary programs have been linked.

    \item[PROVIDES]
      Identifies which shared libraries the package installs into the
      image. This list is used when generating a linkage report.

    \end{Description}

\subsubsection{Kconfig Menu Variables}

    \begin{Description}[DESCRIPTIONXXX]
    \item[CATEGORY]
      Specifies a category for the package. The package will appear on
      the same menu together with the others in the same category.
    \item[DESCRIPTION]
      A short sentence or two describing the package.
    \end{Description}

\subsubsection{Dependencies to Other Packages}

    If your package depends on other packages in order to build or to
    install, you must specify the dependencies using list variables,
    as shown in the following listing. Each dependency has two parts,
    the package name and the build step.

    For example, in the listing below, the variable {\fw BUILD\_DEPEND}
    is set to {\fw basefiles.build}. This means that the build step
    for the current makefile depends on the results of the build step
    of the ``basefiles'' package. Even if your package has no
    dependencies, your makefile must still include the file called
    {\fw depend.mk} at the very end of your makefile.

\begin{lstlisting}
PREP_DEPEND =
BUILD_DEPEND = basefiles.build
INSTALL_DEPEND = basefiles.install
\end{lstlisting}

\subsubsection{Required Targets} \label{RequiredTargets}

    There are seven required targets: fetch, unpack, prep, build,
    install, clean and distclean. You can choose to do nothing for a
    target, if nothing needs to be done. If you create your makefile
    using the {\fw make new\_package} target, as described in
    Section~\ref{MakeNew}, then the generated makefile presents
    reasonable default rules.

\subsubsection{Writing the Rules}

    The \slim build system exports a number of environment variables
    to your makefile, listed below. By convention, variables set by
    the build system are in lower case, and user customizable
    variables are upper case. When referencing a path or a script in
    your makefile, always use the system variables, rather than hard
    coding any paths.

    \begin{Description}[HOSTCCX]
    \item[build]
      The directory to hold the build products.  You should untar your
      sources into this directory.

    % comply - internal use only

    \item[dld]
      The download directory, in which to put the source tar files.

    \item[etc]
      The directory containing the sources for {\fw /etc} on the
      target.

    \item[fetch]
      A script you should use to get your source tar file.

    % gitbase - internal use only
    % liclist - internal use only
    % needed - internal use only
    % redist - internal use only

    \item[rootfs]
      The target root file system. Install the build products into
      this directory. Bear in mind that we are making an embedded
      system, so you might want to pick out just the essential files,
      and not run the normal 'make install'.

    \item[stage]
      The staging area for development libraries and headers. If your
      package provides a shared library for other packages, then do a
      full install into this directory.

    % stamp - internal use only

    \item[start]
      A script you should use to install a startup script for your
      application or daemon.

    \item[unpack]
      A script you may use to unpack your source tar file. If your
      package originates from a third party and is licensed from under
      the GPL, and you have modified it, then you should use this
      script. It will automatically diff the original sources with
      your modified version.

    \item[HOSTCC]
      The host compiler, usually just plain old gcc.

    \item[Q]
      This controls the build verbosity and will be either set to @,
      or it will be empty. Precede each rule that you write with this
      variable.

    \item[ac\_env]
      An environment suitable for passing to a GNU configure script.

    \item[ac\_flags]
      Command line options suitable for passing to a GNU configure
      script.

    \item[karch]
      This variable contains the name of the appropriate Linux
      architecture for the particular board.

    \item[kconfig]
      If this variable is set, then the user would like to run the
      interactive menuconfig option, if any.

    \item[muffle]
      This macro can be used to control the build verbosity. It
      redirects the standard output to {\fw /dev/null}. Add this to
      the end of your rules if your upstream build is too noisy.

    \end{Description}

\subsection{Debugging your makefile}

    \slim considers your makefile to be a prerequisite of your
    package. Every time you edit your makefile, this will cause a
    complete package remake, staring with the ``fetch'' stage.
    The assumption behind this is that when the package's makefile
    changes, the version might have changed, and thus a rebuild is
    required. When developing and debugging a new package, this
    behavior can be counter productive. \slim offers a few special
    targets to work around the default actions in this situation.
    To force a package (in this example, the ``bob'' package) up
    to date, use the following make command line.

\begin{lstlisting}[language=bash,escapechar=X]
X\$X make pkg/bob/update
\end{lstlisting}

    You can run each of the five main targets, namely fetch, unpack,
    prep, build, and install, can be run individually for a particular
    package under development, as shown.

\begin{lstlisting}[language=bash,escapechar=X]
X\$X make pkg/bob/fetch
X\$X make pkg/bob/unpack
X\$X make pkg/bob/prep
X\$X make pkg/bob/build
X\$X make pkg/bob/install
\end{lstlisting}

\section{Adding a New Board}

    Adding a new board type is initially just as easy as adding a new
    package. To add a new board named ``alice,'' enter the following
    command.

\begin{lstlisting}[language=bash,escapechar=X]
X\$X make new_board BRD=alice
\end{lstlisting}

   This will create under {\fw config/alice} a new board derived from
   the generic board. Every new board requires a few first steps.

   \begin{enumerate}
   \item Edit {\fw config/\$BOARD/etc/inittab} to set the desired login terminals.
   \item Run {\fw make menuconfig} to customize the Board Settings menu.
   \end{enumerate}

\section{License Compliance}
% explain license compliance
% explain packages with multiple licenses
\subsection{License Notification}
\subsection{Redistributables}
\subsection{Upstream Sources}
% explain the UPSTREAM variable and the unpack.sh script

\section{Working with \slim}
\subsection{Continuous Disintegration}
% explain baselines
% explain off line development
\subsection{Release Management}

\bibliographystyle{elsarticle-num}
\bibliography{refs}
\end{document}

* support mutiple flavors
  - two or more package sets, possibly overlapping
  - compile shared packages just once
  - stamp directory partly? shared
  - separate download directories (otherwise might mix same package name, two different SHA1 sums)
  - separate rootfs (make repopulate)
  - separate stage (otherwise might mix same library name, two different versions)
  - separate build products:
    1. kernel image
    2. file system image
    3. redist: manifest, licenses, gpl sources
